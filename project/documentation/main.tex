\documentclass{beamer}

\usetheme{Madrid}
\usecolortheme{beaver}
\usepackage{hyperref}
\usepackage{graphicx}
\usepackage{amsmath}
\usepackage{booktabs}
\usepackage{tikz}

\newcommand{\keyword}[1]{\textbf{\color{blue}#1}}

\title[EAN-13 OCR Recognition]{EAN-13 Barcode Recognition \\ An OCR-Based Approach}
\subtitle{Pattern Recognition Systems}
\author{Moldovan Darius-Andrei}
\institute[TUC-N]{Technical University of Cluj-Napoca}
\date{\today}

\begin{document}

% Title slide
\begin{frame}
  \titlepage
\end{frame}

% Outline
\begin{frame}{Outline}
  \tableofcontents
\end{frame}

% Section 1
\section{Introduction}

\begin{frame}{What is an EAN-13 Barcode?}
  \begin{columns}
    \begin{column}{0.6\textwidth}
      \begin{itemize}
        \item \keyword{EAN-13} (European Article Number) consists of 13 digits.
        \item Structure:
        \begin{itemize}
            \item Left Guard (101) 
            \item 6 Left-side digits
            \item Middle Guard (01010)
            \item 6 Right-side digits
            \item Right Guard (101)
        \end{itemize}
        \item \textbf{1 = Bar, 0 = Space}
        \item \textbf{Key Challenge:} While bars encode data geometrically, the human-readable text below provides a robust fallback for recognition.
      \end{itemize}
    \end{column}
    \begin{column}{0.4\textwidth}
      \centering
       \includegraphics[width=\textwidth,height=0.4\textheight]{./images/ean-13-example.png} 
    \end{column}
  \end{columns}
\end{frame}

\begin{frame}{Project Objective}
  \begin{block}{Goal}
    To develop a robust recognition system that identifies EAN-13 barcodes by reading the digits directly using a Convolutional Neural Network (CNN).
  \end{block}

  \begin{itemize}
    \item \keyword{Robustness:} Overcomes issues where scratch marks, printing defects and various lighting conditions may break the vertical bars but leave text readable.
    \item \keyword{Adaptability:} Uses a model trained on diverse fonts to handle real-world variations (Ink bleed, blur).
  \end{itemize}
\end{frame}

% Section 2
\section{Dataset Preparation}

\begin{frame}{Training Data: TMNIST}
  To ensure the model generalizes well to printed text, we utilize the \keyword{TMNIST (Typeface MNIST)} dataset rather than handwritten digits.

  \begin{itemize}
    \item \textbf{Source:} The dataset contains glyphs from over 2,900 different typography fonts.
    \item \textbf{Filtering Process:}
    \begin{itemize}
        \item The original dataset contains 94 characters (letters + symbols).
        \item We extracted and used \textit{only} the digits \textbf{0-9}.
    \end{itemize}
    \item \textbf{Why TMNIST?} 
    \begin{itemize}
        \item Barcodes use standardized fonts (like OCR-B).
        \item Handwritten digits (MNIST) lack the sharp edges and specific geometry of printed barcodes.
        \item TMNIST provides variation in character boldness (simulating ink bleed).
    \end{itemize}
  \end{itemize}
\end{frame}

% Section 3
\section{Preprocessing Pipeline}

\begin{frame}{Pipeline Overview}
  The image processing pipeline is critical for isolating the barcode region before OCR can begin.
  
  \begin{enumerate}
    \item \textbf{Grayscale Conversion:} Reduces computational complexity ($3 \rightarrow 1$ channel).
    \item \textbf{CLAHE:} Adaptive contrast enhancement.
    \item \textbf{Gaussian Blur:} Noise reduction.
    \item \textbf{Edge Detection (Sobel):} Finding vertical structures.
    \item \textbf{Morphology \& Thresholding:} Fusing bars into regions.
  \end{enumerate}
\end{frame}

\begin{frame}{Preprocessing Deep Dive: CLAHE}
  \framesubtitle{Contrast Limited Adaptive Histogram Equalization}
  
  Standard histogram equalization stretches contrast globally, which amplifies noise in flat regions. We use \keyword{CLAHE} to handle uneven lighting (e.g., glare on glossy packaging).

  \begin{columns}
    \begin{column}{0.6\textwidth}
      \textbf{How it works:}
      \begin{enumerate}
          \item \textbf{Tiling:} The image is divided into a grid ($8 \times 8$ tiles).
          \item \textbf{Equalization:} Histogram equalization is applied \textit{locally} to each tile.
          \item \textbf{Contrast Limiting:} To prevent amplifying noise, histogram bins above a threshold are clipped and redistributed.
          \item \textbf{Interpolation:} Boundaries between tiles are smoothed using bilinear interpolation to remove artifacts.
      \end{enumerate}
    \end{column}
    \begin{column}{0.4\textwidth}
      \centering
      \includegraphics[width=\textwidth]{./images/clahe-example.png}
    \end{column}
  \end{columns}
\end{frame}

\begin{frame}{Preprocessing Deep Dive: Sobel Operators}
  \framesubtitle{Why Sobel?}
  
  Barcodes are defined by strong \textbf{vertical gradients}. We use the Sobel operator to detect these specific intensity changes.
  
  \begin{itemize}
      \item The operator uses two $3 \times 3$ convolution kernels:
  \end{itemize}

  \begin{columns}
    \begin{column}{0.4\textwidth}
      \[
      G_x = \begin{bmatrix} 
      -1 & 0 & +1 \\
      -2 & 0 & +2 \\
      -1 & 0 & +1 
      \end{bmatrix}
      \]
      \centering \textit{Detects Vertical Edges}
    \end{column}
    \begin{column}{0.4\textwidth}
      \[
      G_y = \begin{bmatrix} 
      -1 & -2 & -1 \\
      0 & 0 & 0 \\
      +1 & +2 & +1 
      \end{bmatrix}
      \]
      \centering \textit{Detects Horizontal Edges}
    \end{column}
  \end{columns}

  \vspace{0.5cm}
  \textbf{Fusion Strategy:}
  We compute the weighted blend to prioritize vertical structures while retaining rotation robustness:
  \[
  G = 0.5 \cdot |G_x| + 0.5 \cdot |G_y|
  \]
\end{frame}

% Section 4
\section{Region Detection and Recognition}

\begin{frame}{Morphology and Localization}
  After detecting edges, we must group them into a coherent "Barcode Object".

  \begin{enumerate}
      \item \keyword{Otsu's Binarization:} Automatically finds the optimal threshold to separate high-gradient edges from the background.
      \item \keyword{Morphological Closing:} 
      \begin{itemize}
          \item We apply a \textbf{rectangular kernel} ($21 \times 11$).
          \item \textit{Closing = Dilation followed by Erosion.}
          \item \textbf{Purpose:} This specific kernel shape bridges the horizontal gaps between vertical bars, fusing the individual lines into a single solid rectangular block.
      \end{itemize}
      \item \keyword{Erosion \& Dilation:} A secondary pass with a smaller ($3 \times 3$) kernel removes small noise specks (Erosion) and restores the block boundary (Dilation).
  \end{enumerate}
\end{frame}

\begin{frame}{The Neural Network}
  Once the barcode is cropped and the bottom text strip is segmented, the digits are passed to a neural network.
  
  \begin{block}{Architecture Highlights}
    \begin{itemize}
        \item \textbf{Input:} $28 \times 28$ Grayscale image.
        \item \textbf{Conv Layers:} Extract geometric features (loops, lines, corners).
        \item \textbf{Pooling:} Reduces dimensionality and makes the model invariant to small shifts.
        \item \textbf{Dropout:} Prevents overfitting to specific fonts.
        \item \textbf{Output:} Softmax probability over 10 classes (0-9).
    \end{itemize}
  \end{block}
  
  This replaces traditional template matching, allowing the system to read digits even if they are slightly blurred or deformed by perspective.
\end{frame}

% Section 5
\section{Results}

\begin{frame}[c]{Results (1/4)}
  \centering
  \includegraphics[width=0.8\textwidth, height=0.405\textheight]{./images/results/1/1.png}
  \includegraphics[width=0.8\textwidth, height=0.405\textheight]{./images/results/1/2.png}
\end{frame}

\begin{frame}[c]{Results (2/4)}
  \centering
  \includegraphics[width=\textwidth, height=0.81\textheight]{./images/results/1/3.png}
\end{frame}

\begin{frame}[c]{Results (3/4)}
  \centering
  \includegraphics[width=0.8\textwidth, height=0.405\textheight]{./images/results/2/1.png}
  \includegraphics[width=0.8\textwidth, height=0.405\textheight]{./images/results/2/2.png}
\end{frame}

\begin{frame}[c]{Results (4/4)}
  \centering
  \includegraphics[width=\textwidth, height=0.81\textheight]{./images/results/2/3.png}
\end{frame}

\end{document}