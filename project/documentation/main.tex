\documentclass{beamer}

\usetheme{Madrid}
\usecolortheme{seahorse}
\usepackage{hyperref}

\title[EAN-13 Scanning]{EAN-13 Barcode Detection and Recognition using Edge, Hough Transform, and Contour-based Decoding}
\author{Moldovan Darius-Andrei}
\institute[TUC-N]{Technical University of Cluj-Napoca}
\date{November 5, 2025}

\begin{document}

% Title slide
\begin{frame}
  \titlepage
\end{frame}

% Outline
\begin{frame}{Outline}
  \tableofcontents
\end{frame}

% Section 1
\section{Introduction}

\begin{frame}{What is an EAN-13 Barcode?}
  \begin{itemize}
    \item EAN-13 = European Article Number, 13 digits.
    \item Encodes numeric information in sequences of black and white bars.
    \item Commonly used on products in European and Romanian retail.
  \end{itemize}
\end{frame}

\begin{frame}{Project Goal}
  \begin{block}{Objective}
    Develop a robust method to detect and decode an EAN-13 barcode
    directly from its bar pattern using classical image processing.
  \end{block}
  \begin{itemize}
    \item Input: grayscale or color image containing one barcode.
    \item Output: the recognized 13-digit numeric sequence.
    \item Must be invariant to lighting, rotation, distance, and perspective.
  \end{itemize}
\end{frame}

% Section 2
\section{Method Overview}

\begin{frame}{General Pipeline}
  \begin{enumerate}
    \item Image preprocessing.
    \item Edge detection.
    \item Orientation estimation using Hough Transform.
    \item Barcode localization via contours.
    \item Perspective correction and flattening.
    \item Bar pattern decoding (EAN-13 digit extraction).
  \end{enumerate}
\end{frame}

\begin{frame}{1. Image Preprocessing}
  \begin{itemize}
    \item Convert to grayscale and apply \textbf{CLAHE} to improve local contrast.
    \item Apply smoothing to reduce random noise.
    \begin{itemize}
      \item prepares the image for stable edge detection under uneven illumination.
    \end{itemize}
  \end{itemize}
\end{frame}

\begin{frame}{2. Edge Detection}
  \begin{itemize}
    \item Use \textbf{Sobel} filters to find vertical edges.
    \item Threshold to keep only strong vertical transitions.
    \item Perform morphological closing to join fragmented edges.
    \begin{itemize}
      \item barcode zones show dense, repetitive vertical gradients.
    \end{itemize}
  \end{itemize}
\end{frame}

\begin{frame}{3. Orientation Estimation (Hough Transform)}
  \begin{itemize}
    \item Apply \textbf{Hough Line Transform} to detect dominant line angles.
    \item Rotate the image to align the barcode horizontally.
    \begin{itemize}
      \item simplifies contour extraction and decoding.
    \end{itemize}
  \end{itemize}
\end{frame}

\begin{frame}{4. Barcode Region Extraction}
  \begin{itemize}
    \item Find connected regions using \textbf{cv::findContours}.
    \item Filter candidates by aspect ratio (wide/short) and edge density.
    \item Use \textbf{cv::minAreaRect} to compute bounding box.
    \item Apply \textbf{warpPerspective} to correct skew or perspective distortion.
  \end{itemize}
\end{frame}

\begin{frame}{5. Bar Pattern Decoding}
  \begin{itemize}
    \item Scan multiple horizontal lines across the barcode region.
    \item Measure alternating black and white run lengths.
    \item Normalize widths to the module size (smallest bar width).
    \item Convert patterns to binary (0 = narrow, 1 = wide).
    \item Split into left and right halves and decode using EAN-13 tables.
    \item Validate with checksum:
      \[
      \text{Sum} = 3 \times \text{even digits} + \text{odd digits}
      \]
      \[
      \text{Check digit} = (10 - (\text{Sum} \bmod 10)) \bmod 10
      \]
  \end{itemize}
\end{frame}

% Section 3
\section{Improving Robustness}

\begin{frame}{Handling Real-World Variations}
  \begin{itemize}
    \item \textbf{Lighting:} CLAHE and adaptive thresholding handle shadows/reflections.
    \item \textbf{Rotation:} corrected using Hough-detected dominant angle.
    \item \textbf{Distance:} scale invariance through adaptive module normalization.
    \item \textbf{Perspective:} rectified using \textbf{minAreaRect} + \textbf{warpPerspective}.
  \end{itemize}
\end{frame}

\begin{frame}{Validation and Error Checking}
  \begin{itemize}
    \item Validate decoded digits with the EAN-13 checksum.
    \item Compare results from multiple scanlines; choose the most consistent.
    \item Reattempt decoding with adjusted thresholds if checksum fails.
  \end{itemize}
\end{frame}

% Section 4
\section{Results and Discussion}

\begin{frame}{Advantages \& Limitations}
  \begin{block}{Advantages}
    \begin{itemize}
      \item Fully classical, no OCR or ML models required.
      \item Robust to lighting, rotation, and perspective.
      \item Real-time performance on standard hardware.
    \end{itemize}
  \end{block}
  \begin{block}{Limitations}
    \begin{itemize}
      \item Sensitive to blur or low contrast.
      \item Curved surfaces may distort bar widths.
    \end{itemize}
  \end{block}
\end{frame}

% Section 5
\section{Conclusion}

\begin{frame}{Conclusion}
  \begin{itemize}
    \item Edge, Hough, and contour analysis localize the barcode accurately.
    \item Direct decoding of bar patterns retrieves the 13-digit code.
    \item Meets robustness requirements with deterministic, interpretable steps.
  \end{itemize}
  \vspace{0.5cm}
  \textbf{Future work:} improve decoding on curved or blurred barcodes.
\end{frame}

% Section 6
\section{Bibliography}

\begin{frame}{Bibliography and References}
  \begin{itemize}
    \item OpenCV Documentation - Edge Detection (Sobel, Scharr):  
      \url{https://docs.opencv.org/4.x/d2/d2c/tutorial_sobel_derivatives.html}
    \item OpenCV Hough Transform Reference:  
      \url{https://docs.opencv.org/4.x/d3/de6/tutorial_js_houghlines.html}
    \item OpenCV Contour Detection:  
      \url{https://docs.opencv.org/4.x/d4/d73/tutorial_py_contours_begin.html}
    \item ZXing-C++ (EAN-13 decoding logic reference):  
      \url{https://github.com/zxing-cpp/zxing-cpp}
    \item EAN-13 Barcode Specification (GS1):  
      \url{https://www.gs1.org/standards/barcodes/ean-upc}
  \end{itemize}
\end{frame}

\end{document}