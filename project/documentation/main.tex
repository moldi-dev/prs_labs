\documentclass{beamer}

\usetheme{Madrid}
\usecolortheme{seahorse}
\usepackage{hyperref}
\usepackage{graphicx}
\usepackage{amsmath}
\usepackage{booktabs}

% Custom command for emphasizing key terms
\newcommand{\keyword}[1]{\textbf{\color{blue}#1}}

\title[EAN-13 Detection \& Decoding]{EAN-13 Barcode Recognition: \\ A Comparative Approach}
\subtitle{Geometric Decoding vs. Optical Character Recognition (OCR)}
\author{Moldovan Darius-Andrei}
\institute[TUC-N]{Technical University of Cluj-Napoca}
\date{\today}

\begin{document}

% Title slide
\begin{frame}
  \titlepage
\end{frame}

% Outline
\begin{frame}{Outline}
  \tableofcontents
\end{frame}

% Section 1
\section{Introduction}

\begin{frame}{What is an EAN-13 Barcode?}
  \begin{columns}
    \begin{column}{0.6\textwidth}
      \begin{itemize}
        \item \keyword{EAN-13} (European Article Number) consists of 13 digits.
        \item Structure:
        \begin{itemize}
            \item Left Guard (101) 
            \item 6 Left-side digits
            \item Middle Guard (01010)
            \item 6 Right-side digits
            \item Right Guard (101)
        \end{itemize}
        \item \textbf{1 = Bar, 0 = Space}
        \item \textbf{Unique Feature:} The first digit is \textit{implicit}, encoded in the parity pattern of the left-side digits.
      \end{itemize}
    \end{column}
    \begin{column}{0.4\textwidth}
      \centering
      \begin{center}
         \includegraphics[width=\textwidth,height=0.5\textheight]{./images/ean-13-decoded.jpg}
      \end{center}
    \end{column}
  \end{columns}
\end{frame}

\begin{frame}{Project Objective: Dual Approach}
  \begin{block}{Goal}
    To develop and compare two distinct methods for recognizing EAN-13 barcodes from images.
  \end{block}

  \begin{enumerate}
    \item \keyword{Method A: Geometric Pipeline (Implemented)}
    \begin{itemize}
        \item Uses classical computer vision (Edge Detection, Contours).
        \item Decodes the binary bar widths directly (Scanline Analysis).
    \end{itemize}
    
    \item \keyword{Method B: OCR Pipeline (Future Work)}
    \begin{itemize}
        \item Detects the text numbers below the bars.
        \item Uses Tesseract or Deep Learning OCR engines.
    \end{itemize}
  \end{enumerate}

  \textbf{Why two methods?} To evaluate robustness speed vs. accuracy in challenging conditions (blur, curvature, lighting).
\end{frame}

% Section 2
\section{Method A: Geometric Pipeline}

\begin{frame}{Pipeline Overview}
  The primary implemented method follows a strict classical processing chain:
  
  \begin{enumerate}
    \item \textbf{Preprocessing:} Grayscale conversion, CLAHE, Gaussian Blur.
    \item \textbf{Edge Detection:} Sobel gradients to find vertical bar structures.
    \item \textbf{Localization:} Hough Transform (Rotation) + Contour Analysis (Bounding Box).
    \item \textbf{Extraction:} Perspective warping to obtain a flat, rectangular crop.
    \item \keyword{Decoding Algorithm:} Converting pixels into numbers.
  \end{enumerate}
\end{frame}

% Deep Dive into Decoding
\section{Deep Dive: The Decoding Algorithm}

\begin{frame}{Step 1: The Scanline \& Binarization}
  How do we translate an image crop into data?
  
  \begin{itemize}
    \item \keyword{Scanlines:} We extract a single row of pixels (intensity profile) from the center of the cropped barcode.
    \item \keyword{Robustness:} If the center fails, we scan offsets ($\pm 10$ pixels) to avoid defects or glare.
    \item \keyword{Adaptive Binarization:} The line is converted to 0s (Spaces) and 1s (Bars) based on local intensity thresholds.
  \end{itemize}
  
  \[
  \text{Pixel Stream: } [255, 255, 0, 0, 0, 255, 255, 0, 255] \rightarrow [0, 0, 1, 1, 1, 0, 0, 1, 0]
  \]
\end{frame}

\begin{frame}{Step 2: Run-Length Encoding (RLE)}
  We do not process individual pixels; we process \keyword{relative widths}.
  
  \begin{itemize}
    \item We count consecutive pixels of the same color.
    \item \textbf{Example:}
    \begin{itemize}
        \item \textit{Binary:} 0000 11 000 11111 00
        \item \textit{RLE:} [4, 2, 3, 5, 2]
    \end{itemize}
    \item This creates a sequence of "Widths" representing Bars and Spaces.
    \item \textbf{Finding the Start:} We look for the Guard Pattern (1-1-1 ratio) to identify the beginning of the data.
  \end{itemize}
\end{frame}

\begin{frame}{Step 3: Normalization (The Core Logic)}
  A digit in EAN-13 always consists of \textbf{4 elements} (Space-Bar-Space-Bar) and has a total width of \textbf{7 modules}.
  
  \begin{block}{The Normalization Formula}
    To handle zoom and resolution differences, we normalize the raw RLE values ($P_i$) to the standard module width:
    \[
    M_i = \text{round}\left( \frac{P_i}{\sum_{j=1}^{4} P_j} \times 7 \right)
    \]
  \end{block}
  
  \begin{example}
    Raw RLE for one digit: $[12px, 8px, 4px, 4px]$ (Sum = 28px). \\
    Normalized: $\frac{12}{28}\times7 \approx 3$, $\frac{8}{28}\times7 \approx 2$, ... \\
    \textbf{Result Pattern:} 3-2-1-1 $\rightarrow$ Lookup Table $\rightarrow$ Digit \textbf{0}.
  \end{example}
\end{frame}

\begin{frame}{Step 4: Decoding and Parity}
  \begin{columns}
    \begin{column}{0.5\textwidth}
      \textbf{Left Side Encoding:}
      \begin{itemize}
          \item Digits 0-9 have two encodings: \keyword{L-code} (Odd parity) and \keyword{G-code} (Even parity).
          \item The scanner detects which code was used (L or G).
      \end{itemize}
    \end{column}
    \begin{column}{0.5\textwidth}
      \textbf{The 13th Digit (First Digit):}
      \begin{itemize}
          \item The first digit is \textit{not} printed as bars.
          \item It is determined by the sequence of L and G codes found in the first 6 digits.
          \item E.g., $LLGLGG \rightarrow 1$.
      \end{itemize}
    \end{column}
  \end{columns}
  
  \vspace{0.5cm}
  \textbf{Right Side:} Always uses R-codes.
\end{frame}

% Section 3
\section{Method B: OCR Pipeline (Future Work)}

\begin{frame}{The OCR Approach}
  Instead of analyzing bar widths, we analyze the human-readable text below the bars.
  
  \begin{block}{Pipeline Plan}
    \begin{enumerate}
        \item \textbf{Text Region Proposal:} Use morphological operations (dilation) to merge text characters into a solid block.
        \item \textbf{Extraction:} Crop the bottom 20\% of the detected barcode area.
        \item \textbf{Recognition:} Feed the crop into Tesseract LSTM or a CNN trained on digits (MNIST).
    \end{enumerate}
  \end{block}
\end{frame}

% Section 4
\section{Comparative Analysis Plan}

\begin{frame}{Expected Comparison}
  \begin{table}[]
    \centering
    \begin{tabular}{@{} l l l @{}}
      \toprule
      \textbf{Feature} & \textbf{Method A (Geometric)} & \textbf{Method B (OCR)} \\
      \midrule
      \textbf{Speed} & Extremely Fast ($<10$ms) & Slower (NN inference) \\
      \textbf{Blur} & Fails if edges merge & Often robust (context) \\
      \textbf{Lighting} & Needs contrast & Needs contrast \\
      \textbf{Curvature} & Complex to de-warp & Robust (deformable models) \\
      \textbf{Damage} & Redundancy helps & Fails if text is scratched \\
      \bottomrule
    \end{tabular}
    \caption{Hypothetical performance matrix}
  \end{table}
  
  The final project will perform benchmarks on a dataset of real-world images to validate these hypotheses.
\end{frame}

\end{document}